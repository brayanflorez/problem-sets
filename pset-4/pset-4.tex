% Options for packages loaded elsewhere
\PassOptionsToPackage{unicode}{hyperref}
\PassOptionsToPackage{hyphens}{url}
%
\documentclass[
]{article}
\usepackage{amsmath,amssymb}
\usepackage{iftex}
\ifPDFTeX
  \usepackage[T1]{fontenc}
  \usepackage[utf8]{inputenc}
  \usepackage{textcomp} % provide euro and other symbols
\else % if luatex or xetex
  \usepackage{unicode-math} % this also loads fontspec
  \defaultfontfeatures{Scale=MatchLowercase}
  \defaultfontfeatures[\rmfamily]{Ligatures=TeX,Scale=1}
\fi
\usepackage{lmodern}
\ifPDFTeX\else
  % xetex/luatex font selection
\fi
% Use upquote if available, for straight quotes in verbatim environments
\IfFileExists{upquote.sty}{\usepackage{upquote}}{}
\IfFileExists{microtype.sty}{% use microtype if available
  \usepackage[]{microtype}
  \UseMicrotypeSet[protrusion]{basicmath} % disable protrusion for tt fonts
}{}
\makeatletter
\@ifundefined{KOMAClassName}{% if non-KOMA class
  \IfFileExists{parskip.sty}{%
    \usepackage{parskip}
  }{% else
    \setlength{\parindent}{0pt}
    \setlength{\parskip}{6pt plus 2pt minus 1pt}}
}{% if KOMA class
  \KOMAoptions{parskip=half}}
\makeatother
\usepackage{xcolor}
\usepackage[margin=1in]{geometry}
\usepackage{graphicx}
\makeatletter
\def\maxwidth{\ifdim\Gin@nat@width>\linewidth\linewidth\else\Gin@nat@width\fi}
\def\maxheight{\ifdim\Gin@nat@height>\textheight\textheight\else\Gin@nat@height\fi}
\makeatother
% Scale images if necessary, so that they will not overflow the page
% margins by default, and it is still possible to overwrite the defaults
% using explicit options in \includegraphics[width, height, ...]{}
\setkeys{Gin}{width=\maxwidth,height=\maxheight,keepaspectratio}
% Set default figure placement to htbp
\makeatletter
\def\fps@figure{htbp}
\makeatother
\setlength{\emergencystretch}{3em} % prevent overfull lines
\providecommand{\tightlist}{%
  \setlength{\itemsep}{0pt}\setlength{\parskip}{0pt}}
\setcounter{secnumdepth}{-\maxdimen} % remove section numbering
\ifLuaTeX
\usepackage[bidi=basic]{babel}
\else
\usepackage[bidi=default]{babel}
\fi
\babelprovide[main,import]{spanish}
% get rid of language-specific shorthands (see #6817):
\let\LanguageShortHands\languageshorthands
\def\languageshorthands#1{}
\usepackage{xcolor}
\usepackage[colorlinks = true, linkcolor = blue, urlcolor  = blue, citecolor = blue, anchorcolor = blue]{hyperref}
\newcommand{\MYhref}[3][blue]{\href{#2}{\color{#1}{#3}}}
\ifLuaTeX
  \usepackage{selnolig}  % disable illegal ligatures
\fi
\usepackage{bookmark}
\IfFileExists{xurl.sty}{\usepackage{xurl}}{} % add URL line breaks if available
\urlstyle{same}
\hypersetup{
  pdftitle={Taller de R: Estadística y Programación},
  pdfauthor={Problem set 4},
  pdflang={es},
  hidelinks,
  pdfcreator={LaTeX via pandoc}}

\title{Taller de R: Estadística y Programación}
\author{Problem set 4}
\date{2024-04-20}

\begin{document}
\maketitle

En este taller se evalúan los temas vistos en las clases 11 y 13 del
curso. Lea atentamente las instrucciones del taller.

\begin{center} \textcolor{blue}{\subsection{Instrucciones}} \end{center}

\begin{itemize}
\item
  Este taller representa el \textbf{25\%} de la nota total del curso y
  puede ser realizado de manera individual o en grupos de hasta 3
  personas. En las primeras líneas del script, escriba su nombre, código
  y la versión de R que está utilizando. Además, al inicio del código,
  debe incluir las librerías que utilizará en la sesión, por ejemplo:
  \texttt{pacman}, \texttt{rio}, \texttt{data.table},
  \texttt{tidyverse}, \texttt{sf}, y \texttt{rvest}.
\item
  Asegúrese de descargar las bases de datos desde el repositorio
  \url{https://github.com/taller-r-202403/problem-sets} y de crear un
  nuevo repositorio en su cuenta de GitHub. Si trabaja en grupo, solo un
  integrante debe crear el repositorio y compartir el acceso con los
  demás. El repositorio debe ser público para permitir el acceso desde
  cualquier cuenta de GitHub. Incluya al menos tres carpetas en el
  repositorio: \texttt{input} (datos originales), \texttt{output} (datos
  procesados), y \texttt{code} (script con la respuesta del taller).
\item
  Todos los integrantes del grupo deben publicar el enlace al
  repositorio de GitHub en la actividad \textbf{Problem-set-3} del
  Bloque Neón antes de las 23:59 horas del 26 de abril de 2024.
\item
  Por favor, organice su trabajo cuidadosamente y comente paso a paso
  cada línea de código. Recuerde \textbf{NO} usar acentos ni caracteres
  especiales dentro del código para evitar problemas al abrir los
  scripts en diferentes sistemas operativos.
\item
  No seguir estas instrucciones resultará en una penalización del
  \textbf{20\%} en la nota final.
\end{itemize}

\begin{center} \textcolor{blue}{\subsection{Solucionar:}} \end{center}

Se debe emplear un bucle o una función para importar los archivos de la
Gran Encuesta Integrada de Hogares que se encuentran en la carpeta
\texttt{input}. Además, si es necesario, se puede acceder al
\href{https://microdatos.dane.gov.co/index.php/catalog/782/data-dictionary}{\color{blue}{diccionario}}
o descargar el documento
\href{https://www.dane.gov.co/files/operaciones/GEIH/bol-GEIH-dic2023.pdf}{\color{blue}{técnico}}
de los datos.

\subsection{\texorpdfstring{\textbf{1. Bucle
(35\%)}}{1. Bucle (35\%)}}\label{bucle-35}

\begin{itemize}
\tightlist
\item
  \textbf{1.1} \textbf{Lista de archivos} \texttt{input}
\end{itemize}

Cree un objeto que almacene el vector de nombres de los archivos dentro
de la carpeta \texttt{input}. Asegúrese de que cada archivo contenga la
ruta con la ubicación de cada archivo.

\textbf{Hint:} Para este punto, puede usar la función
\texttt{list.files()} con el argumento \texttt{recursive\ =\ T}.

\begin{itemize}
\tightlist
\item
  \textbf{1.2} \textbf{Importar archivos:}
\end{itemize}

Usa el objeto creado en el punto anterior como insumo de una función que
permita importar los archivos de \texttt{Fuerza\ de\ trabajo},
\texttt{No\ ocupados} y \texttt{Ocupados} para todos los meses.

\textbf{Hint:} Para este punto, puedes crear una función que importe un
archivo y combinarla con la función \texttt{lapply}. O puedes crear un
bucle y almacenar los dataframes en un objeto tipo lista.

\begin{itemize}
\tightlist
\item
  \textbf{1.3 Combinar conjuntos de datos}
\end{itemize}

Combina todos los \texttt{data.frame} que importaste en el punto
anterior tres \texttt{data.frame}.

\textbf{Hint:} Para este punto, puede usar la función \texttt{rbindlist}
de la librería \texttt{data.table}.

\subsection{\texorpdfstring{\textbf{2. Preparación
(35\%)}}{2. Preparación (35\%)}}\label{preparaciuxf3n-35}

\begin{itemize}
\tightlist
\item
  \textbf{2.1 Creación d bases de datos}
\end{itemize}

Cree tres bases de datos diferentes:

\begin{enumerate}
\def\labelenumi{\arabic{enumi}.}
\item
  Usando la base de datos llamada \texttt{fuerza\ de\ trabajo}, suma el
  número de individuos que hacen parte de la fuerza laboral (\textbf{ft
  == 1}) y aquellos que hacen parte de la población en edad de trabajar
  (\textbf{pet == 1}) por mes, asegurándote de tener en cuenta el factor
  de expansión.
\item
  Usando la base de datos llamada \texttt{Ocupados}, suma el número de
  individuos que se encuentren empleados (\textbf{ft == 1}) por mes,
  asegurándote de tener en cuenta el factor de expansión.
\item
  Usando la base de datos llamada \texttt{No\ ocupados}, suma el número
  de individuos desempleados (\textbf{dsi == 1}) por mes, asegurándote
  de tener en cuenta el factor de expansión.''
\end{enumerate}

\begin{itemize}
\tightlist
\item
  \textbf{2.2 Colapsar datos a nivel mensual}
\end{itemize}

Unifica todas las bases de datos creadas en el punto anterior en una
única base llamada \texttt{Output}, que debe contener al menos cinco
columnas: \texttt{Población\ en\ edad\ de\ trabajar},
\texttt{fuerza\ laboral}, \texttt{ocupados}, \texttt{desempleados} y el
\texttt{mes} correspondiente.

\begin{itemize}
\tightlist
\item
  \textbf{2.3 Tasas de desempleo y ocupación.}
\end{itemize}

Divida el número de individuos \texttt{desempleados} por la
\texttt{fuerza\ laboral} para obtener la tasa de desempleo, y los
\texttt{ocupados} por la \texttt{población\ en\ edad} de trabajar para
obtener la tasa de ocupación.

\subsection{\texorpdfstring{\textbf{3. GGplot2
(30\%)}}{3. GGplot2 (30\%)}}\label{ggplot2-30}

Grafique las tasas de desempleo y ocupación para cada mes utilizando la
función \texttt{geom\_line}. (\textbf{Hint:} Realice un pivot wider a
las tasas de modo que los valores estén en una sola columna)

\end{document}
